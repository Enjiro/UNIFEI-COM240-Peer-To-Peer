\documentclass[a4paper]{article}

% Pacotes para o português.
\usepackage[brazilian]{babel}
\usepackage[utf8]{inputenc}
\usepackage[T1]{fontenc}

\usepackage{sbc-template}

\usepackage{datetime}
\usepackage{graphicx}
\usepackage{float}
\usepackage{indentfirst}
\usepackage{amssymb}
\usepackage{multirow}
\usepackage[nottoc]{tocbibind}

% Espaçamento duplo.
\usepackage{setspace}
 % Breaklines nas citações.
\usepackage{cite}

% Linha horizontal.
\newcommand{\HRule}{\rule{\linewidth}{0.5mm}}
\newcommand{\hRule}{\rule{4.5cm}{0.1mm}}

% Listagens.
\newfloat{program}{thp}{lop}
\floatname{program}{Listagem}

% Figuras.
\newcommand{\figurex}[5]
{
\begin{figure}[htb, h!]
   	\setlength{\unitlength}{1.0cm}
   	\centering
   	\includegraphics[scale=#1]{./img/#2.#3}
	\begin{center}
	   	\parbox{.9\linewidth}{\footnotesize \sf \caption{#5}  \label{#4}}
	\end{center}
\end{figure}
}

\newcommand{\ck}[0]
{
\checkmark
}

\hyphenation{}

\begin{document}

%\doublespacing
\onehalfspacing

\begin{titlepage}
\begin{center}

% Topo 1.
\textsc{\Large UNIVERSIDADE FEDERAL DE ITAJUBÁ\\
	INSTITUTO DE MATEMÁTICA E COMPUTAÇÃO}\\[0.7cm]

% Topo 2.
\textsc{DEPARTAMENTO DE MATEMÁTICA E COMPUTAÇÃO}\\[2.8cm]

% Título.
\textsc{\Large Seminário}\\
\HRule \\[0.4cm]
{\Large \bfseries Arquitetura Peer to Peer}
\HRule \\[0.4cm]
\textsc{REDES DE COMPUTADORES}\\[2.8cm]

% Etc.
\begin{minipage}{0.4\textwidth}
\begin{flushleft} \large
\emph{Alunos:}\\[0.43cm]
%\hRule\\
David Mateus Batista\\
Gabriel Erzinger Dousseau\\
Gabriel Alves Taets\\
Mauricio Leite\\
\end{flushleft}
\end{minipage}
\begin{minipage}{0.4\textwidth}
\begin{flushright} \large
\emph{Professor:}\\[0.4cm]
%\hRule\\
Bruno Guazzelli Batista
\end{flushright}
\end{minipage}

\vfill

\today

\end{center}
\end{titlepage}

\thispagestyle{empty}

\section*{Resumo}
Esta monografia tem o objetivo de estudar e analisar a arquitetura de redes Peer-to-peer,
suas vantagens, desvantagens, limitações e aplicações.
\newpage
\pagenumbering{Roman}

\tableofcontents
\newpage
\pagenumbering{arabic}

%http://www.fio.edu.br/manualtcc/co/1_Estrutura_do_TCC.html%
\section{Introdução}
\newpage

\section{Fundamentação Teórica}
\newpage

\section{Discussão}
\newpage

\section{Considerações finais}
\newpage

%*********************************************************************************************
\newpage
\bibliographystyle{apalike}
\bibliography{bibliography}

\end{document}

